Emergency department (ED) crowding is a global public health issue that has been repeatedly associated with increased mortality. Predicting future service demand would enable preventative measures aiming to eliminate crowding along with it's detrimental effects. Recent findings in our ED indicate that occupancy ratios exceeding 90\% are associated with increased 10-day mortality. In this paper, we aim to predict these crisis periods using retrospective time series data such as weather, availability of hospital beds, calendar variables and occupancy statistics from a large Nordic ED with a LightGBM model. We predict mortality associated crowding for the whole ED and individually for it's different operational sections. We demonstrate that afternoon crowding can be predicted at 11 a.m. with an AUC of 0.82 (95\% CI 0.78-0.86) and at 8 a.m. with an AUC up to 0.79 (95\% CI 0.75-0.83). Consequently we show that forecasting mortality-associated crowding using time series data is feasible.