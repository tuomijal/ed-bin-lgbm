In this study, we were set out to investigate whether mortality associated ED crowding can be predicted with sufficient margin for action. Our results suggest that 1) forecasting mortality-associated crowding is feasible using anonymous administrative data, 2) LightGBM model demonstrates high predictive accuracy, particularly for medical and bedoccupying patients and achieves an AUC of 0.80 by 11 a.m, 3) sections differ from one another in terms of both prevalence of crowding and predictability and 4) predicting is possible without access to patient level data which makes the model implementable regardless of the current privacy regulation. The study highlights the need for integrating these forecasts with actionable interventions to enhance patient safety and optimize ED operations. 

We identify several directions for future studies. The performance and utility of the model has to be validated in a prospective setting and in close collaboration with the clinical stakeholders, although creating a continuous integration to hospital information infrastructure will be a formidable task. A prospective study demonstrating the impact of the enabled interventions is also required. Future research should also explore the benefit of more granular data which could for example indicate the combined severity of the currently occupied patients or the aggregated required level of care.