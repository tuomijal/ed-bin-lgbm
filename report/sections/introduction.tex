Emergency department crowding remains a persistent problem worldwide, the adverse effects of which include increased number of medication errors \cite{Kulstad2010}, delays in onset of medication such as antibiotics \cite{Pines2007}, analgesics \cite{Pines2008} or thrombolysis \cite{Schull2003}, prolonged length of stay \cite{McCarthy2009} and dissatisfaction of the patients \cite{Boudreaux2000, Sun2000}. Most importantly, however, crowding has been repeatedly associated with increased mortality in several countries including Australia \cite{Richardson2006}, Canada \cite{Guttmann2011}, China \cite{Zhang2019}, United States \cite{Sun2013}, Sweden \cite{Ugglas2021}, South Korea \cite{Jo2014}, United Kingdom \cite{Jones2022} and Finland \cite{Eidsto2023}. There are multiple temporal and local reasons for a single crowding event, but the recurrent, global and worsening nature of the phenomenon point to a more systemic underlying cause: aging populations and difficulty to hire the required medical personnel to satisfy the ever increasing demand \cite{Morley2018}. In Europe, as populations have been projected to age up until year 2100 \cite{eurostat}, there is no quick and spontaneous resolution of the problem in sight.

For these reasons, there has been a continued interest in optimizing the utilization of the limited resources that are readily available. One manifestation of this effort is emergency department forecasting that has been of academic interest since 1981 \cite{Diehl1981} with over 100 articles published ever since \cite{Gul2018}. Over the last few years, machine learning has taken over traditional statistical models in the field of time series forecasting \cite{Makridakis2022, Makridakis2020} and this been reflected in ED forecasting as well. For example, we recently demonstrated the superiority of LightGBM over both traditional statistical and novel deep learning models \cite{Tuominen2024} and the performance of ensemble models has been investigated by others as well \cite{Petsis2022, Alvarez2023}. Some tailored deep learning architectures have also been proposed for example using convolutional neural networks \cite{Sharafat2021}, variable autoencoders \cite{Harrou2020}, and generative adversarial networks \cite{Kadri2023} while some more traditional statistical learning approaches such as INGARCH models are still being investigated \cite{Reboredo2023}. We have also prospectively evaluated the performance of an early warning software for crowding \cite{Tuominen2023}. Despite these contributions, several important gaps in the literature remain.

First, vast majority of previous studies have focused on predicting patient volumes or occupancy in continuous terms \cite{Gul2018}. This is an obvious and intuitive approach, but it has two problems: the resulting metrics are difficult to communicate to an ED stakeholder and impossible to transfer to other facilities. Continuous error metrics such as the mean absolute error or root mean squared error are useful when models are compared against each other with the same sample. But they are not useful in answering the question that really matters: is this performance accurate enough for deployment in the clinic? As colleagues \citet{Hoot2009} so elegantly put it in 2009:

\begin{quote}
    An early warning system must incorporate two components: 1) a clear definition of a crisis period, and 2) a means of predicting crises.
\end{quote}

For the last 15 years, the first -- and the most important -- part of this wisdom seems to have been forgotten, excluding few notable exceptions \cite{Xie2022}. The definition of a crisis period is important because it is the pre-requisite for action. After all, forecasting is never an end in itself but a means to an end: regardless of the context it always aims to enable pre-emptive maneuvers that prevent an adverse outcome in the future. So what is this adverse outcome or \emph{a clear definition of a crisis period} in the context of an ED? We believe the answer to be self-evident: the ED is in crisis when patient safety becomes compromised due to the crowded state alone. There is increasing evidence that associate certain emergency department occupancy ratios (EDOR) with increased mortality. Most of them have simply compared the most crowded quartile with the less crowded ones \cite{Richardson2006, Jo2014}, which serves to document the existence of the association, but does not reveal the specific threshold for mortality associated crowding. We recently showed that an EDOR exceeding 90\% is associated with increased 10-day mortality in a large Nordic combined ED \cite{Eidsto2023}. In this study, we aim to predict these periods of mortality associated crowding and ultimately enable administrative interventions that would prevent them.

Second, the majority of previous effort has focused on forecasting aggregated visit statistics such as total arrivals or occupancy which is a natural first step, especially if the focus of the work is in developing algorithms because it helps to limit the computational complexity of the work. However, aggregated visit statistics can be misleading. For example, the Nordic combined ED under the investigation in this article consists of several largely independent sections with their dedicated personnel. A section can be crowded without directly affecting the others and for example crowding of the medical section without crowding of the surgical section is common. Moreover, the implications of crowding vary significantly depending on the respective section. For example, the crowding of the medical section, where elder patients with more comorbidities are common, is likely far more dangerous than crowding of the walk-in section where most patients present with lower acuity conditions. Thus, it is important to account for this heterogeneity when forecasting the future. Surprisingly, we are aware of only one study by \citet{Aroua2015} in which this was accounted for by using diagnosis-based stratification.

Third, the access to health data has become more stringent because of the emerging national and European regulation. While this carries benefits for ensuring privacy, extensive regulation has constraining effects for both innovation and research as has been recently demonstrated \cite{peukert2022, bessen2020, Bruck2023}. In fact, the current legislation makes simulation-based approaches -- as proposed e.g. by \citet{Hoot2008} -- difficult if not impossible to implement. For this reason, the models in this study have been designed to operate with anonymous administrative time series data.

In this study, we aim to address these gaps by developing an early warning algorithm for ED crowding that i) predicts mortality associated crowding; ii) performs stratified predictions for operational sections of the ED; iii) does not require access to personal information to operate; and iv) provides sufficient temporal margin for action. 